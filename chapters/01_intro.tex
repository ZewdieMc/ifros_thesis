\chapter{Introduction}
\label{ch:intro}

% - General description of robotics;
% - General description of perception;
% - General description of localization.

Unlike many other LiDAR odometry systems such as ICP-based methods, FAST-LIO employs a direct lidar-inertial(LiDAR-IMU) fusion approach. IMU inputs are used to predict the state, and LiDAR input are used to update the state and map.

% order of sentences should be arranged for coherence
USING ICP TO DIRECTLY REGISTER 3D LiDAR POINTS TO MAP RESULTS IN EXTREMELY POOR RESULTS \cite{Blanco-Claraco-RSS-19}. This is because of the uneven distribution of 3D LiDAR points where the dense near points dominate the cost function of ICP optimization.

\textbf{Loop closure:} The process of recognizing that the robot has returned to its previously visited location. This is important for correcting errors in the map and the robot's estimated trajectory. Without loop closure, even small localization errors accumulate over time causing the map to drift and be inaccurate. \textbf{Wrong loops} and their effects in causing divergence in the map should also be explained.
\textbf{Reverse direction loop} and \textbf{the normal loop possibly ....circular loop}(need to clearly explain this).
% evaluation to enable future use
% document structure
\section{Background and motivation}
\section{Problem Statement}
\section{Objectives}
\section{Thesis structure}


% research questions
% motivation/justification
% scope, limitations
